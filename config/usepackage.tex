%Petr - pouzivam

%obecne
\usepackage[czech]{babel}
\usepackage[utf8]{inputenc}
\usepackage[IL2]{fontenc}
% \usepackage{charter}

\usepackage{framed}
\usepackage{mathtools}
\usepackage{pdflscape}
\usepackage[top=3cm, left=3.5cm, right=2.5cm, bottom=3cm, headheight=15pt, includeheadfoot]{geometry}%rozměry stránky
\usepackage{textcomp}
\usepackage[round]{natbib}
\usepackage{hyperref}
\hypersetup{
    bookmarks=true,         % show bookmarks bar?
    unicode=false,          % non-Latin characters in Acrobat’s bookmarks
    pdftoolbar=true,        % show Acrobat’s toolbar?
    pdfmenubar=true,        % show Acrobat’s menu?
    pdffitwindow=false,     % window fit to page when opened
    pdfstartview={FitH},    % fits the width of the page to the window
    pdftitle={Smoderp manual},    % title
    pdfauthor={Kavka ...},     % author
    pdfsubject={Subject},   % subject of the document
    pdfcreator={Creator},   % creator of the document
    pdfproducer={Producer}, % producer of the document
    pdfkeywords={keyword1, key2, key3}, % list of keywords
    pdfnewwindow=true,      % links in new PDF window
    colorlinks=false,       % false: boxed links; true: colored links
    linkcolor=red,          % color of internal links (change box color with linkbordercolor)
    citecolor=green,        % color of links to bibliography
    filecolor=magenta,      % color of file links
    urlcolor=cyan           % color of external links
}
\usepackage[printonlyused]{acronym} %rejstik
% \usepackage{acronym} %rejstik
\makeatletter
\AtBeginDocument{%
  \renewcommand*{\AC@hyperlink}[2]{#2}%
}
\makeatother



%text
\usepackage{subcaption}
\usepackage{caption}
\usepackage{listings} % inserting code 

%tabulky
\usepackage{array}
\usepackage{tabulary}
\usepackage{multirow}
\usepackage{multicol}

%obrazky









%Lenka - nepouzivam
\usepackage{longtable}
\usepackage{siunitx}
\usepackage{rotating}
\usepackage[table,xcdraw]{xcolor}
\usepackage{booktabs}
\usepackage{url}
%\usepackage[pdftex,unicode,bookmarksnumbered,raiselinks=true]{hyperref}
\usepackage{indentfirst}
\usepackage{fancyhdr}
\usepackage[font={footnotesize},labelfont=bf,justification=centering]{caption}
\usepackage{hhline}
\usepackage{colortbl}
\usepackage{array,graphicx}
\usepackage{placeins}

\usepackage{titlesec}
\setcounter{secnumdepth}{4}
% 
\titleformat{\paragraph}
{\normalfont\normalsize\bfseries}{\theparagraph}{1em}{}
\titlespacing*{\paragraph}
{0pt}{3.25ex plus 1ex minus .2ex}{1.5ex plus .2ex}


\usepackage[table]{xcolor}
\usepackage{setspace}


% vetsi mezera mezi odstavci
\setlength{\parskip}{0.5em}


% nadefinovane tvary do flow chart diagramu
% ten jeden po postaven v ./graph/CZflowch
\usepackage{tikz}
\usetikzlibrary{shapes.geometric, arrows}
\tikzstyle{startstop} = [rectangle, rounded corners, minimum width=3cm, minimum height=1cm,text centered, draw=black, fill=red!30]
\tikzstyle{io} = [trapezium, trapezium left angle=70, trapezium right angle=110, minimum width=1cm, minimum height=1cm, text centered, draw=black, fill=blue!30]
\tikzstyle{arrow} = [thick,->,>=stealth]
\tikzstyle{decision} =  [diamond, minimum width=1cm, minimum height=1cm, text centered, text width=2cm, draw=black, fill=green!30, aspect=2]
\tikzstyle{process} = [rectangle, minimum width=3cm, minimum height=1cm, text centered, text width=3cm, draw=black, fill=orange!30]
\tikzstyle{guide} = [inner sep=0pt,minimum size=0mm]
\tikzstyle{line} = [thick]



\usepackage{dirtree}
\usepackage{indentfirst}

% tohle je jen prikaz na delatni popisu rovnic 
% jeho pouziti he v kapitole pouzite vztahy treba
\newcommand{\jj}[2]{
   & \acs{#1} & je \acl{#1}#2 \\
}


