% 11.12. Zakomentova radky jsou to tempu


\begin{table}[h!]
 

 \centering
 \caption{Přehled rastrových výstupů}
\label{tab:vystupyrast}

% \begin{tabular}{p{4cm}lp{2cm}p{5cm}}
 \begin{tabular}{llp{0.5\textwidth}}
 \hline
  Název souboru    & Jednotka    & Popis       \\ 
  (ESRI nebo .acs)    &     &        \\ \hline \hline
  cInfiltrationM     &   $m$        & Kumulativní infiltrace \\
  cRainfallM       &  $m$  &  Kumulativní srážka (bez intercepce a povrchové retence) \\
  cVolInM3       &  $m^3$  & Kumulativní objem přítoků do buňky  (plošný + rýhový) \\
  cSheetVolOutM3       &  $m^3$  & Kumulativní objem odtoku z buňky \\
  cRillVolOutM3       &  $m^3$  & Kumulativní objem odtoku z buňky rýhou \\
  cVolRestM3      &  $m^3$  & Kumulativní zbytek po odtoku odtoku z buňky\\
  mSurfaceFlowM3\_S  &   $m^3-s^{-1}$	&  Maximální celkový v buňce\\
  mVelocityM\_S	&   $ms^{-1}$	&  Maximální rychlost proudění v buňce (plošného či soustředěného odtoku) \\
  mReachFlowM3\_S   &  $NA$ & doplnym  \\
  mShearStressPa   & $Pa$ &  tečné napětí \\
  reachFID   &  $NA$ &  Buňky mimo tok mají {\tt NODATA} hodnotu nebo id daného úseku toku.  \\
  massBalance   &   $m$  &  Bilance všech vstupů a výstupu z a do buňky  \\
  \hline
%  tyhle jdou do tempu
%  rillAreaM       &   $m^2$      &  Plocha buňky, kterou porývá rýha \\
%  finalCellState    &  $NA$ & Typ odtoku buňky na konci výpočtu (viz sekce~\ref{sec:statpopis})\\
%  critWaterLevelM         & $m$    &  kritický výška hladiny \\
%  mSheetFlowM3_S	  &   $m^3-s^{-1}$	&  Maximální plošný průtok v buňce  \\
%  mRillFlowM3_S    &   $m^3-s^{-1}$	&  Maximální soustředěný průtok v buňce\\
%  mSheetWaterLevelM    &   $m$  &   Maximální výška hladiny plošného v buňce \\
%  mRillWaterLevelM   &   $m$  &   Maximální výška hladiny soustředěného odtoku v buňce \\

 \end{tabular}
 

\end{table}


% cInfiltrationM
% cRainfallM
% cSheetVolOutM3 
% cRillVolOutM3
% cSurfaceVolOutM3
% cVolRestM3
% cReachVolOutM3
% mSurfaceFlowM3_S
% mVelocityM_S
% mReachFlowM3_S
% mShearStressPa
% reachFID
% massBalance



