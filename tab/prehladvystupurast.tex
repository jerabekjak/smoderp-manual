% 11.12. Zakomentova radky jsou to tempu


\begin{table}[t]
 

 \centering
 \caption{Přehled rastrových výstupů}
\label{tab:vystupyrast}

% \begin{tabular}{p{4cm}lp{2cm}p{5cm}}
 \begin{tabular}{llp{0.5\textwidth}}
 \hline
  Název souboru    & Jednotka    & Popis       \\ 
  (ESRI nebo .acs)    &     &        \\ \hline \hline
  cinfilt\_m      &   $m$        & Kumulativní infiltrace \\
  crainf\_m          &  $m$    &  Kumulativní srážka (bez intercepce a povrchové retence) \\
%   csheetvout\_m3      &  $m^3$  & Kumulativní objem odtoku z buňky \\
%   crillvout\_m3       &  $m^3$  & Kumulativní objem odtoku z buňky rýhou \\
  csurvout\_m3       &  $m^3$  & Kumulativní objem odtoku z buňky \\
%   cvin\_m3            &  $m^3$  & Kumulativní objem přítoků do buňky  (plošný + soustředěný) \\
  volrest\_m3          &  $m^3$  & Objem vody zbylé v buňkách po zkončení výpočtu\\
  dmt                 &  $m$ &  Výřez použitého digitálního modelu terénu \\
  flowdir             &  $NA$ &  Rastr s uloženými směry odtoku  \\
  mshearstr\_pa      & $Pa$ &  Maximální tečné napětí \\
%   mreachflm3\_s     &  $m^3s^{-1}$ & Maximální odtok v úsecích hyd. sítě  \\
  msurfl\_m3\_s   &   $m^3s^{-1}$ &  Maximální celkový odtok v buňce\\
  mvel\_m\_s       &   $ms^{-1}$ &  Maximální rychlost proudění v buňce (plošného či soustředěného odtoku) \\
  reachFID            &  $NA$  &  Označuje úseky toku (=fid + 1000), buňky s plošným odtokem (=0) a plošným i soustředěným odtokem (=1) \\
  massbalance         &   $m$  &  Bilance všech vstupů a výstupu z a do buňky  \\
  \hline
%  tyhle jdou do tempu
%  rillAreaM       &   $m^2$      &  Plocha buňky, kterou porývá rýha \\
%  finalCellState    &  $NA$ & Typ odtoku buňky na konci výpočtu (viz sekce~\ref{sec:statpopis})\\
%  critWaterLevelM         & $m$    &  kritický výška hladiny \\
%  mSheetFlowM3_S	  &   $m^3-s^{-1}$	&  Maximální plošný průtok v buňce  \\
%  mRillFlowM3_S    &   $m^3-s^{-1}$	&  Maximální soustředěný průtok v buňce\\
%  mSheetWaterLevelM    &   $m$  &   Maximální výška hladiny plošného v buňce \\
%  mRillWaterLevelM   &   $m$  &   Maximální výška hladiny soustředěného odtoku v buňce \\

 \end{tabular}
 

\end{table}


% cInfiltrationM
% cRainfallM
% cSheetVolOutM3 
% cRillVolOutM3
% cSurfaceVolOutM3
% cVolRestM3
% cReachVolOutM3
% mSurfaceFlowM3_S
% mVelocityM_S
% mReachFlowM3_S
% mShearStressPa
% reachFID
% massBalance



