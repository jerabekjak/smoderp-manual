%!TEX ROOT = ../main.tex


%\clearpage
%\newpage\null\thispagestyle{empty}\newpage

Dostává se Vám do ruky uživatelský manuál modelu \smod. Celý názvem modelu je: Simulační Model Povrchového Odtoku a Erozního Procesu. Tento model lze využít pro výpočet hydrologicko erozních procesů na jednotlivých pozemcích nebo na malých povodích. Výstupy z modelu jsou primárně určeny pro stanovení odtokových poměrů v ploše povodí a parametrů opatření pro snížení odtoku z povodí a erozního ohrožení zemědělské půdy. Model lze využít při navrhování komplexnějších soustav sběrných a odváděcích prvků nebo suchých nádrží a polderů. Jeho využití předpokládají jak současné metodiky, tak i technické normy a doporučené standardy.
Z hlediska kategorizace modelu se jedná o fyzikálně založený plně distribuovaný dvourozměrný model epizodní model. Nově zavedené prostorové řešení (2D), které nahradilo dřívější profilovou verzi modelu, umožňuje komplexní řešení a náhled na celou řešenou lokalitu. Dvourozměrné řešení je z hlediska vstupních dat a vnitřních procesů složitější, nicméně benefity distribuovaného řešení převažují. Dostupnost vstupních dat v podrobném rozlišení se zlepšuje, stejně tak jako se zvyšuje výpočetní kapacita výpočetní techniky.
Vývoj modelu je podporován z veřejných prostředků a podílejí se na něm studenti a zaměstnanci Katedry hydromeliorací a krajinného inženýrství Fakulty stavební ČVUT v Praze
Pro snazší orientaci je manuál je rozdělen na tři základní části. V první části jsou uvedeny výpočtové vztahy a popis jednotlivých zvolených procesů. Druhá část je věnována vstupním a výstupním datům a je zde stručně popsán tok programu. V třetí části jsou ukázány výsledky při řečení konkrétní lokality.
Případné aktualizace modelu, vzorová data, ukázky využití a další informace jsou pak průběžně poskytovány na stránkách  modelu
(\href{http://storm.fsv.cvut.cz/cinnost-katedry/volne-stazitelne-vysledky/smoderp/?lang=cz}{storm.fsv.cvut.cz/cinnost-katedry/volne-stazitelne-vysledky/smoderp/}).