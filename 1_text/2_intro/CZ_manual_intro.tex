%!TEX ROOT = ../main.tex


%\clearpage
%\newpage\null\thispagestyle{empty}\newpage

Dostává se Vám do ruky uživatelský manuál k modelu SMODERP2D. Model se celým názvem původně jmenoval Simulační Model Povrchového Odtoku a Erozního Procesu. Tento model lze využít pro výpočet hydrologicko erozních procesů na jednotlivých pozemcích nebo na malých povodích. Výstupy z modelu jsou primárně určeny pro stanovení odtokových poměrů v ploše povodí a pro stanovení případných opatření pro snížení odtoku z povodí a pro snížení erozního ohrožení zemědělské půdy. Model neslouží jen pro navrhování konkrétních prvků na jednotlivých pozemcích, ale lze jej využít i při navrhování komplikovanějších soustav sběrných a odváděcích prvků. Dále je možné tento model využít i pro navrhování suchých nádrží a poldrů. Jeho využití předpokládají jak současné metodiky, tak i technické normy a doporučené standardy.
Z hlediska kategorizace modelu se jedná o fyzikálně založený plně distribuovaný dvourozměrný model sloužící pro simulace konkrétních srážkových epizod. Nově zavedené prostorové řešení (2D), které nahradilo dřívější profilovou verzi modelu, umožňuje komplexní řešení a náhled na celou řešenou lokalitu. Z hlediska vstupních dat a vnitřních procesů se jedná sice o složitější variantu řešení. Nic méně benefity prostorového řešení převažují. Dostupnost vstupních dat v podrobném rozlišení se zlepšuje, stejně tak jako se zvyšuje výpočetní kapacita výpočetní techniky.
Vývoj modelu je podporován z veřejných prostředků a podílejí se na něm studenti a zaměstnanci Katedry hydromeliorací a krajinného inženýrství Fakulty stavební ČVUT v Praze
Pro snazší orientaci je manuál je rozdělen na dvě základní části. V první části jsou uvedeny výpočtové vztahy a popis jednotlivých zvolených procesů. V druhá část je pak věnována vstupním datům a výstupům.
Případné aktualizace modelu, vzorová data, ukázky využití a další informace jsou pak průběžně poskytovány na stránkách  modelu
(\href{http://storm.fsv.cvut.cz/cinnost-katedry/volne-stazitelne-vysledky/smoderp/?lang=cz}{storm.fsv.cvut.cz/cinnost-katedry/volne-stazitelne-vysledky/smoderp/}).