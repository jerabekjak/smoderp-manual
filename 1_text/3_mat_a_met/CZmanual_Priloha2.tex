Při přípravě vstupních dat, může docházet k problémům, jejichž příčiny jsou ze základních výstupů těžko odhalitelné nebo samotné výpisy základních výsledků modlu  můžou být základní vypsaná data pro danou aplikaci nedostatečná. Při běhu preprocessingu jsou ukládány dočasné vrstvy do adresářů {\tt temp} a {\tt temp\_dp} (pokud jsou řešeny i úseky hydrografické sítě). Tyto adresáře ve výchozím nastavení modelu na konci výpočtu smazány. Při běhu programu lze obdržet detailnější informaci o řešených procesech než je ukázány a základních výstupech. Pokud uživatel narazí na jeden z výše zmíněných problému má možnost pomocí dalšího parametru modelu zamezit mazání adresářů {\tt temp} a {\tt temp\_dp} a mít k dispozici další výstupní rastry a veličiny vypsaná v hydrogramech. Při tomto postupu je ale třeba zasáhnout do jednoho ze zdrojových skriptů a proto se doporučuje uživatelům, kteří budou \smod spouštět přímo ze zdrojového kód (pokud je \smod nainstalovaný pomocí spustitelného instalačního souboru, není tento postup možný). 