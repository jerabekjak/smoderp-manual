%!TEX ROOT = ../mainCZ.tex







Výstupy modelu jsou uloženy složky zadané jako vstupní parametr. Složka při spuštěné programu vyčistí. Výstupy jsou ve textových a rastrový formátech. Jako rastrový formát lze zvolit proprietární ESRI formát nebo formát ASII. Přehled rasterových výstupních souborů je v tabulce\ref{tab:vystupyrast}. Pokud jsou do vstupů zadány body pro výpočet hydrogramů vypíší se do souborů s příponou {\tt.dat}. Soubory mají hlavičku podle procesů, které v buňce pod daným bodem vzniknou. Popis vypsaných veličin je popsán tabulce~\ref{tab:vystupnidat}

% 11.12. Zakomentova radky jsou to tempu


\begin{table}[t]
 

 \centering
 \caption{Přehled rastrových výstupů}
\label{tab:vystupyrast}

% \begin{tabular}{p{4cm}lp{2cm}p{5cm}}
 \begin{tabular}{llp{0.5\textwidth}}
 \hline \hline
  Název souboru    & Jednotka    & Popis       \\ 
  (ESRI nebo .acs)    &     &        \\ \hline 
  cinfilt\_m      &   $m$        & Kumulativní infiltrace \\
  crainf\_m          &  $m$    &  Kumulativní srážka (bez intercepce a povrchové retence) \\
%   csheetvout\_m3      &  $m^3$  & Kumulativní objem odtoku z buňky \\
%   crillvout\_m3       &  $m^3$  & Kumulativní objem odtoku z buňky rýhou \\
  csurvout\_m3       &  $m^3$  & Kumulativní objem odtoku z buňky \\
%   cvin\_m3            &  $m^3$  & Kumulativní objem přítoků do buňky  (plošný + soustředěný) \\
  volrest\_m3          &  $m^3$  & Objem vody zbylé v buňkách po zkončení výpočtu\\
  dmt                 &  $m$ &  Výřez použitého digitálního modelu terénu \\
  flowdir             &  $NA$ &  Rastr s uloženými směry odtoku  \\
  mshearstr\_pa      & $Pa$ &  Maximální tečné napětí \\
%   mreachflm3\_s     &  $m^3s^{-1}$ & Maximální odtok v úsecích hyd. sítě  \\
  msurfl\_m3\_s   &   $m^3s^{-1}$ &  Maximální celkový odtok v buňce\\
  mvel\_m\_s       &   $ms^{-1}$ &  Maximální rychlost proudění v buňce (plošného či soustředěného odtoku) \\
  reachFID            &  $NA$  &  Označuje úseky toku (=fid + 1000), buňky s plošným odtokem (=0) a plošným i soustředěným odtokem (=1) \\
  massbalance         &   $m$  &  Bilance všech vstupů a výstupu z a do buňky  \\
  \hline \hline
%  tyhle jdou do tempu
%  rillAreaM       &   $m^2$      &  Plocha buňky, kterou porývá rýha \\
%  finalCellState    &  $NA$ & Typ odtoku buňky na konci výpočtu (viz sekce~\ref{sec:statpopis})\\
%  critWaterLevelM         & $m$    &  kritický výška hladiny \\
%  mSheetFlowM3_S	  &   $m^3-s^{-1}$	&  Maximální plošný průtok v buňce  \\
%  mRillFlowM3_S    &   $m^3-s^{-1}$	&  Maximální soustředěný průtok v buňce\\
%  mSheetWaterLevelM    &   $m$  &   Maximální výška hladiny plošného v buňce \\
%  mRillWaterLevelM   &   $m$  &   Maximální výška hladiny soustředěného odtoku v buňce \\

 \end{tabular}
 

\end{table}


% cInfiltrationM
% cRainfallM
% cSheetVolOutM3 
% cRillVolOutM3
% cSurfaceVolOutM3
% cVolRestM3
% cReachVolOutM3
% mSurfaceFlowM3_S
% mVelocityM_S
% mReachFlowM3_S
% mShearStressPa
% reachFID
% massBalance






% zakomentovane je do tmp

\begin{table}[h!]
 

 \centering
 \caption{Popis veličin  v {\tt.dat} souborech}
\label{tab:vystupydat}

% \begin{tabular}{p{4cm}lp{2cm}p{5cm}}
 \begin{tabular}{llp{0.5\textwidth}}
  \hline  \hline
 Název sloupce    & Jednotka    & Popis       \\ 
  \hline
%  \hline
%  Buňka s plošným odtokem:	 &&\\
 time[s]          &   $s$      &  Čas výpočetního kroku          \\
 deltaTime[s]     &   $s$        &  Aktuální délka časového kroku  \\
 rainfall[m]      &  $m$         &  Srážková výška v aktuálním časovém kroku \\
%  sheetWaterLevel[m]       &  $m^3$  & Výška hladiny plošného odtoku \\
%  sheetFlow[m3/s]       &  $m^3s^{-1}$  & Průtok plošného odtoku  \\
%  sheetVolRunoff[m3]    &  $m^3$     & Odteklý objem plošného odtoku \\
%  sheetVolRest[m3]      &  $m^3$     & Objem zbytku vody po plošném odtoku \\
%  infiltration[m]         &  $m$      & Výška infiltrace v daném časovém kroku \\
%  surfaceRetention[m]    &  $m$      & Výška zadržené vody na povrchu v daném časovém kroku \\
%  callState                   &  -         & Typ odtoku na buňce (viz sekce~\ref{sec:statpopis})  \\
%  inflowVol[m3]   &   $m^3$ &  Celkový objem přítoku do buňky \\
 totalWaterLevel[m]	  &   $m$	&  Celková výška hladiny  \\ 
%  \hline
%  Pro soustředěný odtok &&\\ \hline 
%  rillWaterLevel[m]         &   $m$       &  Výška hladiny v buňce se soustředěným odtokem* \\
%  rillWidth[m]	       &   $m$ &  Šířka rýhy vzniklá soustředěným odtokem\\
%  rillFlow[m3/s]      &   $m^3s^{-1}$       &  Průtok v rýze soustředěného odtoku \\
%  rillVolRunoff[m3]   &   $m^3$  &   Objem soustředěného odtoku rýhou \\
%  rillVolRest[m3]  &  $m^3$ &   Objem zbytku vody po soustředěném odtoku rýhou  \\
 surfaceFlow[m3/s]   &  $m^3_{-1}$ & Celkový průtok (plošný + soustředěný)  \\
 surfaceVolRunoff[m3]   &   $m$  & Celkový odteklý objem (plošný + soustředěný) \\
%  rillInflowVol[m3] & $m$ &  @@@ toto tam chcem? to je V\_inflow cast co jde do ryhy, pridal jsem to tam jednou kdyz jsem hledal nejakou chybu...\\
%  ratio & $m$ &  Počet krácení časového kroku v rýhách @@@(je pro nas?)\\
%  sheetCourantCrit & $m$ &  Courantovo kritérium pro plošná odtok @@@(je pro nas?)\\
%  rillCourantCrit & $m$ & Courantovo kritérium pro soustředěný odtok @@@(je pro nas?) \\
%  nIter & $m$ &  Počet iterací pří výpočty daného výpočetního kroku @@@(to bych tam nechal, muže to napověděl jestli se tam neděje něco moc rychle, což může znamenat chybu v zadaných datech, třeba dát 600 mm do srážky místo 60 mm) \\
  \hline
   \hline
   \multicolumn{3}{p{\textwidth}}{*výška hladiny u soustředěného odtoku není výška skutečné výška hladiny v rýze, ale v nadkritická výška hladiny vztažená na celou plochu výpočetní buňky}
 \end{tabular}

\end{table}





%\section{Výstupní data} \label{section:vystupnidata}
\textbf{Zde dodelat}
\begin{itemize}
\item popsat výstupy mimo temp
\item popsat co jsou v temp
\item popsat výstupy v určitých krocích
\end{itemize}
%Po úspěšném ukončení modelu je do výstupního adresáře uloženo několik souborů. Každý z těchto souborů obsahuje hodnoty pro každou buňku rastru. Buňky, na kterých neprobíhal výpočet neobsahují žádné hodnoty, tedy NoData. Základní výstupy jsou uvedeny přímo ve zvoleném výstupním adresáři. Mimo hlavní výstupy jsou volitelně ukládány i dočasné výstupy sloužící pro případnou kontrolu. V podadresáři \textbf{temp} jsou dočasné soubory výpočtu v ploše a v podadresáři \textbf{temp_dp} jsou dočasné soubory vodních toků. \textbf{Dočasným výsledkům bude věnována jedna z dalších kapitol}
\textbf{tenhle seznam doplnit popisem o co tam jde a v jakých je to jednotkách}
\begin{itemize}
\item VRestEndRillL3.asc
\item TotalBil.asc
\item toky.asc
\item SurRet.asc
\item stream.shp
\item ShearStress.asc
\item body hydrogamů (*.dat) - průběh veličin pro jednotlivé body zadané v kapitole XXXX
\item MaxWaterRillL.asc
\item MaxWateL.asc
\item MaxVelovity.asc
%\item MaxQRillL3t_1.asc
%\item MaxQL3t_1.asc
%\item HCrit.asc
%\item FinalState.asc
\item CumVRestL3.asc
\item CumVOutRillL3.asc
\item CumVOutL3.asc
\item CumVInL3.asc
\item AreaRill.asc
\end{itemize}

\textbf{toto je origoš z DP}

\par Ne vždy se vytvoří všechny tyto výstupní soubory. Záleží na zvolených vstupních parametrech. Pokud uživatel nezadá žádnou bodovou vrstvu, nevytvoří se poslední textový soubor. 
V případě, že uživatel nezvolí možnost soustředěného odtoku, nevytvoří se rastry a shapefile související s tímto typem odtoku. Rastr soustředění odtoku se nevytvoří při nezvolení vícesměrného odtoku. 
Ostatní soubory se vytvoří pokaždé.  

\textbf{ z diplomky}

Výstupy se ukládají do adresáře nazvaného output. Cestu k němu si volí uživatel v rámci vstupních dat (viz kap. 2.3.1). Model prochází stále vývojem a dotýká se to i výstupních souborů. Princip ale zůstává stejný a jedná se spíše o úpravy zdrojového kódu zajištující lepší přehlednost a práci s kódem pro budoucí úpravy. Např. práce s vícerozměrnými maticemi a převedení všech výpočtů do základních (SI) jednotek. 
Výsledkem modelu jsou soubory (.shp, .rst, .txt, .dbf), které reprezentují parametry (Zajíček J., 2014):
hladina
Výstupem jsou hodnoty maximální výšky hladiny pro každou buňku. Jedná se tedy o rastrovou vrstvu vytvořenou porovnáváním hodnot výšek hladiny v každém časovém kroku. Uložena je nejvyšší hodnota. Výška hladiny v jednotlivých krocích je získána pomocí bilance přítoků a odtoků do buňky.  
průtok
Výstupem jsou hodnoty maximálního průtoku pro každou buňku. Obdobně jako u hladiny jsou porovnávány hodnoty v jednotlivých krocích a uložena maximální hodnota. Hodnoty průtoku v jednotlivých časových krocích jsou vypočteny pomocí metody kinematické vlny (teorie viz kap. 1.5.2).
infiltrace
Výstupem infiltrace jsou hodnoty v každé buňce, které jsou během doby běhu modelu postupně načítány až do vyčerpání infiltrační kapacity.
zbytkový objem
Zbytkovým objemem se rozumí objem, který v dané buňce v časovém kroku zůstal. V případě odtoku veškeré vody z rastru je rastr nulový. Matematicky je objem vyjádřen jako rozdíl celkového objemu v buňce (zbytkový objem z předchozího kroku a přítoky) a povrchového a soustředěného odtoku.
odtok
Výstup týkající se odtoku slouží pro konečnou bilanci (kontrolu) a testování. Jedná se o celkové množství, které z buňky odteklo za celou dobu běhu modelu. 
rychlost
Rastr rychlostí je výstupem sloužící k určení erozní ohroženosti. Porovnávány jsou hodnoty skutečných rychlostí s limitními nevymílacími rychlostmi (viz tab. č. 3 ).
napětí. 

Obdobou je rastr tečného napětí. Slouží k určení míst potencionálně nebezpečných. Hodnoty limitních hodnot tečného napětí jsou uvedeny ve stejné tabulce jako rychlosti průtok v rýze (viz tab. č. 3 ).

Průtok v rýze je rastrová vrstva znázorňující maximální průtok v rýze při soustředěném odtoku. Výstup je vytvořen jen při volbě typu výpočtu s uvažováním rýhového odtoku. Rýha vznikne pouze v buňkách, kde výška hladiny překročí hladinu kritickou. 
rychlost v rýze
Rastr obsahuje hodnoty maximální rychlosti v buňkách, kde je rýha vytvořena. Výpočet v rýhách probíhá odlišně oproti povrchovému odtoku. Jedná se o větší rychlosti, a proto na těchto buňkách probíhá výpočet za běžný časový krok 3x. V jiném případě by hrozilo, že výpočet nebude konvergovat.
souhrn

Final evalution.txt je textový soubor, který obsahuje souhrn zadaných vstupů a čas běhu modelu a bilanci vody. 
hydrogram
Point hydrographs.txt je textový soubor s hodnotami výšky hladiny, průtoku, napětí, rychlostí v bodech zadaných vstupní bodovou vrstvou. Soubor slouží k tvorbě hydrogramů v těchto bodech. Automaticky je k vrstvě přidán bod, ve kterém je hodnota flow acumulation nejvyšší.
Výstupem v současnosti je i řada dalších vrstev, které slouží ale spíše k tvorbě a testování modelu a pro samotného uživatele nejsou potřebné.	


\paragraph{Hydrogramy} \label{sec:hydrogramy}



