
\pozn{
\textbf{Zde dodelat}
\begin{itemize}
  \item popsat výstupy mimo temp
  \item popsat co jsou v temp
  \item popsat výstupy v určitých krocích
\end{itemize}
}


Výstupy modelu jsou uloženy do složky zadané mezi vstupními parametry (obsah složky je při spuštěné programu vymazán!). Kumulativní nebo maximální hodnoty v jednotlivých  buňkách jsou na konci výpočtu uloženy v rastrovém formátu (viz kapitola~\ref{sec:rastr}). Průnik polygonů prostorové distribuce typu půd a využití území jsou uloženy ve vektorovém formátu (viz kapitola~\ref{sec:vektor}). Pokud model \smod počítá i úseky hydrografické sítě, jsou kumulativní nebo maximální hodnoty veličin jednotlivých úseků vypsáný v atributové tabulce vektorové vrstvy úseků (viz kapitola~\ref{sec:vektor}), prostorové rozložení jednotlivých úseků je uloženo také jako jeden s rastrů (viz kapitola~\ref{sec:rastr}).  Volitelné výstupy hydrogramů  v bodech jsou ve formě časových řad uloženy do textových souborů s příponou {\tt.dat} (viz kapitola~\ref{sec:hydrogramy}). Další nadstandardní výstupy lze získat způsobem popsaným v příloze . Jednotlivé výstupy jsou dále popsány podrobněji. 













\subsection{Rastrové výstupy}\label{sec:rastr}

V rastrech jsou uloženy vybrané veličiny jednotlivých buňkách řešeného území. Jako rastrový formát lze zvolit proprietární ESRI formát nebo textový formát ASII. Přehled rastrových výstupních souborů je shrnut v tabulce~\ref{tab:vystupyrast}. Pokud jsou v modelu řešeny i úseky hydrografické sítě jsou buňky rastru ležící na úseku uloženy s hodnotou {\tt NoData} (výjimku tvoří 2 rastry popisující vlastnosti úseků, viz tabulka~\ref{tab:vystupyrast}).  


% jsou ve textovém nebo rastrovém formátu.  








\subsection{Vektorové výstupy}\label{sec:vektor}

Výstupní data modelu ve vektorovém formátu jsou tři. Jedná se topologicky upravenou vrstvu úseků hydrografické sítě ({\tt hydReach}), kde jsou do její atributové tabulky doplněny kumulativní a maximální hodnoty vybraných veličin. Tyto veličiny jsou popsány v tabulce~\ref{tab:useky}. Druhým vektorovým výstupem je vrstva, která zobrazuje průnik prostorového rozložení typu půdy a využití území ({\tt interSoilLU}). Ukázka takové vrstvy je na obrázku~\ref{fig:soillu}. Tato vektorová vrstva slouží především ke kontrole správnosti přípravy vstupních dat či hledání chyb v nich. Při preprocessingu jsou  z (nepovinné) bodové vrstvy pro zápis hydrogramů smazány body, které jsou mimo výpočetní oblast. Proto je ve výsledcích uložena vrstva s body, které jsou skutečné pro výpis hydrogramů použity. Tato bodová vrstva má název {\tt pointsCheck}. 













\subsection{Hydrogramy}\label{sec:hydrogramy}

Pokud jsou do vstupů zadány body pro výpis hydrogramů, vypíší se do textových souborů s příponou {\tt.dat}. Vypsané veličiny jsou závislé na typu odtokového procesu. Popis vypsaných veličin při povrchovém odtoku je shrnut v tabulce~\ref{tab:vystupydat}. Pokud je v buňce úsek hydrografické sítě, vypisují se hodnoty tohoto celého úseku, přestože bod není na konci úseku.  Názvy a význam veličin popisující úsek toku jsou popsány v tabulce~\ref{tab:vystupytokdat}. \pozn{věta spis do prvni casi: } Model v současné verzi uvažuje, že pokud je v buňce úsek hydrografické sítě, zabírá úsek celou buňku, přesto že je jeho šířka menší než šířka samotné buňky.  Název těchto souborů je odvozen z FID upravené bodové vrstvy {\tt pointsCheck} ve tvaru $point${{\tt pointsCheck}:FID}$.dat$. 





% 11.12. Zakomentova radky jsou to tempu


\begin{table}[t]
 

 \centering
 \caption{Přehled rastrových výstupů}
\label{tab:vystupyrast}

% \begin{tabular}{p{4cm}lp{2cm}p{5cm}}
 \begin{tabular}{llp{0.5\textwidth}}
 \hline \hline
  Název souboru    & Jednotka    & Popis       \\ 
  (ESRI nebo .acs)    &     &        \\ \hline 
  cinfilt\_m      &   $m$        & Kumulativní infiltrace \\
  crainf\_m          &  $m$    &  Kumulativní srážka (bez intercepce a povrchové retence) \\
%   csheetvout\_m3      &  $m^3$  & Kumulativní objem odtoku z buňky \\
%   crillvout\_m3       &  $m^3$  & Kumulativní objem odtoku z buňky rýhou \\
  csurvout\_m3       &  $m^3$  & Kumulativní objem odtoku z buňky \\
%   cvin\_m3            &  $m^3$  & Kumulativní objem přítoků do buňky  (plošný + soustředěný) \\
  volrest\_m3          &  $m^3$  & Objem vody zbylé v buňkách po zkončení výpočtu\\
  dmt                 &  $m$ &  Výřez použitého digitálního modelu terénu \\
  flowdir             &  $NA$ &  Rastr s uloženými směry odtoku  \\
  mshearstr\_pa      & $Pa$ &  Maximální tečné napětí \\
%   mreachflm3\_s     &  $m^3s^{-1}$ & Maximální odtok v úsecích hyd. sítě  \\
  msurfl\_m3\_s   &   $m^3s^{-1}$ &  Maximální celkový odtok v buňce\\
  mvel\_m\_s       &   $ms^{-1}$ &  Maximální rychlost proudění v buňce (plošného či soustředěného odtoku) \\
  reachFID            &  $NA$  &  Označuje úseky toku (=fid + 1000), buňky s plošným odtokem (=0) a plošným i soustředěným odtokem (=1) \\
  massbalance         &   $m$  &  Bilance všech vstupů a výstupu z a do buňky  \\
  \hline \hline
%  tyhle jdou do tempu
%  rillAreaM       &   $m^2$      &  Plocha buňky, kterou porývá rýha \\
%  finalCellState    &  $NA$ & Typ odtoku buňky na konci výpočtu (viz sekce~\ref{sec:statpopis})\\
%  critWaterLevelM         & $m$    &  kritický výška hladiny \\
%  mSheetFlowM3_S	  &   $m^3-s^{-1}$	&  Maximální plošný průtok v buňce  \\
%  mRillFlowM3_S    &   $m^3-s^{-1}$	&  Maximální soustředěný průtok v buňce\\
%  mSheetWaterLevelM    &   $m$  &   Maximální výška hladiny plošného v buňce \\
%  mRillWaterLevelM   &   $m$  &   Maximální výška hladiny soustředěného odtoku v buňce \\

 \end{tabular}
 

\end{table}


% cInfiltrationM
% cRainfallM
% cSheetVolOutM3 
% cRillVolOutM3
% cSurfaceVolOutM3
% cVolRestM3
% cReachVolOutM3
% mSurfaceFlowM3_S
% mVelocityM_S
% mReachFlowM3_S
% mShearStressPa
% reachFID
% massBalance





% zakomentovane je do tmp




\begin{table}[h!]
 

 \centering
 \caption{Popis veličin  tabulky úseků hydrografické sítě}
\label{tab:useky}

% \begin{tabular}{p{4cm}lp{2cm}p{5cm}}
 \begin{tabular}{llp{0.5\textwidth}}
  \hline  \hline
 Název sloupce        & Jednotka     & Popis                                 \\ 
 \hline
 FID            &   ---          &  Identifikátor přiřazeného úseku ({\it feature id})   \\
 cVolM3         &  $m^3$         & Kumulativní objem odtoku                              \\
 mFlowM3\_S      &   $m^3s_{-1}$  & Maximální průtok                                      \\
 mFlowTimeS     &   $s$          &  Čas dosažení maximálního průtoku                     \\
 mWatLM         &  $m$       &  Maximální výška hladiny v úseku                               \\
%  mWatLTimeS     &  $s$       &  Čas dosažení maximálního průtoku                              \\
%  cSurInFM3      &  $m^3$     &  Kumulativní přítok do úseku z jeho okolí   \\
 restVolM3      &  $m^3$     &  Objem v úseku po skončení výpočtu         \\
%  cVolDomM3      &  $m^3$     &  Kumulativní odtok úseku z řešeného území (hodnota je nulová pokud úsek odtéká do jiného navazujícího úseku)   \\
 toFID          &  ---       &  FID úseku do které daný úsek odtéká (hodnota -9999 vyjadřuje situaci, kdy úsek kříží hranici řešeného a odtéká tedy mimo toto území) \\
  \hline
   \hline
 \end{tabular}

\end{table}



% FID;V_out_cum [L^3];Q_max [L^3.t^{-1}];timeQ_max[s];h_max [L];timeh_max[s];
% Cumulatice_inflow_from_field[L^3];Left_after_last_time_step[L^3];Out_form_domain[L^3];to_reach



% % zakomentovane je do tmp

\begin{table}[h!]
 

 \centering
 \caption{Popis veličin  v {\tt.dat} souborech}
\label{tab:vystupydat}

% \begin{tabular}{p{4cm}lp{2cm}p{5cm}}
 \begin{tabular}{llp{0.5\textwidth}}
  \hline  \hline
 Název sloupce    & Jednotka    & Popis       \\ 
  \hline
%  \hline
%  Buňka s plošným odtokem:	 &&\\
 time[s]          &   $s$      &  Čas výpočetního kroku          \\
 deltaTime[s]     &   $s$        &  Aktuální délka časového kroku  \\
 rainfall[m]      &  $m$         &  Srážková výška v aktuálním časovém kroku \\
%  sheetWaterLevel[m]       &  $m^3$  & Výška hladiny plošného odtoku \\
%  sheetFlow[m3/s]       &  $m^3s^{-1}$  & Průtok plošného odtoku  \\
%  sheetVolRunoff[m3]    &  $m^3$     & Odteklý objem plošného odtoku \\
%  sheetVolRest[m3]      &  $m^3$     & Objem zbytku vody po plošném odtoku \\
%  infiltration[m]         &  $m$      & Výška infiltrace v daném časovém kroku \\
%  surfaceRetention[m]    &  $m$      & Výška zadržené vody na povrchu v daném časovém kroku \\
%  callState                   &  -         & Typ odtoku na buňce (viz sekce~\ref{sec:statpopis})  \\
%  inflowVol[m3]   &   $m^3$ &  Celkový objem přítoku do buňky \\
 totalWaterLevel[m]	  &   $m$	&  Celková výška hladiny  \\ 
%  \hline
%  Pro soustředěný odtok &&\\ \hline 
%  rillWaterLevel[m]         &   $m$       &  Výška hladiny v buňce se soustředěným odtokem* \\
%  rillWidth[m]	       &   $m$ &  Šířka rýhy vzniklá soustředěným odtokem\\
%  rillFlow[m3/s]      &   $m^3s^{-1}$       &  Průtok v rýze soustředěného odtoku \\
%  rillVolRunoff[m3]   &   $m^3$  &   Objem soustředěného odtoku rýhou \\
%  rillVolRest[m3]  &  $m^3$ &   Objem zbytku vody po soustředěném odtoku rýhou  \\
 surfaceFlow[m3/s]   &  $m^3_{-1}$ & Celkový průtok (plošný + soustředěný)  \\
 surfaceVolRunoff[m3]   &   $m$  & Celkový odteklý objem (plošný + soustředěný) \\
%  rillInflowVol[m3] & $m$ &  @@@ toto tam chcem? to je V\_inflow cast co jde do ryhy, pridal jsem to tam jednou kdyz jsem hledal nejakou chybu...\\
%  ratio & $m$ &  Počet krácení časového kroku v rýhách @@@(je pro nas?)\\
%  sheetCourantCrit & $m$ &  Courantovo kritérium pro plošná odtok @@@(je pro nas?)\\
%  rillCourantCrit & $m$ & Courantovo kritérium pro soustředěný odtok @@@(je pro nas?) \\
%  nIter & $m$ &  Počet iterací pří výpočty daného výpočetního kroku @@@(to bych tam nechal, muže to napověděl jestli se tam neděje něco moc rychle, což může znamenat chybu v zadaných datech, třeba dát 600 mm do srážky místo 60 mm) \\
  \hline
   \hline
   \multicolumn{3}{p{\textwidth}}{*výška hladiny u soustředěného odtoku není výška skutečné výška hladiny v rýze, ale v nadkritická výška hladiny vztažená na celou plochu výpočetní buňky}
 \end{tabular}

\end{table}

% zakomentovane je do tmp

\begin{table}[h!]
 

 \centering
 \caption{Popis veličin  v {\tt.dat} souborech}
\label{tab:vystupydat}

% \begin{tabular}{p{4cm}lp{2cm}p{5cm}}
 \begin{tabular}{llp{0.5\textwidth}}
  \hline  \hline
 Název sloupce    & Jednotka    & Popis       \\ 
  \hline
%  \hline
%  Buňka s plošným odtokem:	 &&\\
 time[s]          &   $s$      &  Čas výpočetního kroku          \\
 deltaTime[s]     &   $s$        &  Aktuální délka časového kroku  \\
 rainfall[m]      &  $m$         &  Srážková výška v aktuálním časovém kroku \\
%  sheetWaterLevel[m]       &  $m^3$  & Výška hladiny plošného odtoku \\
%  sheetFlow[m3/s]       &  $m^3s^{-1}$  & Průtok plošného odtoku  \\
%  sheetVolRunoff[m3]    &  $m^3$     & Odteklý objem plošného odtoku \\
%  sheetVolRest[m3]      &  $m^3$     & Objem zbytku vody po plošném odtoku \\
%  infiltration[m]         &  $m$      & Výška infiltrace v daném časovém kroku \\
%  surfaceRetention[m]    &  $m$      & Výška zadržené vody na povrchu v daném časovém kroku \\
%  callState                   &  -         & Typ odtoku na buňce (viz sekce~\ref{sec:statpopis})  \\
%  inflowVol[m3]   &   $m^3$ &  Celkový objem přítoku do buňky \\
 totalWaterLevel[m]	  &   $m$	&  Celková výška hladiny  \\ 
%  \hline
%  Pro soustředěný odtok &&\\ \hline 
%  rillWaterLevel[m]         &   $m$       &  Výška hladiny v buňce se soustředěným odtokem* \\
%  rillWidth[m]	       &   $m$ &  Šířka rýhy vzniklá soustředěným odtokem\\
%  rillFlow[m3/s]      &   $m^3s^{-1}$       &  Průtok v rýze soustředěného odtoku \\
%  rillVolRunoff[m3]   &   $m^3$  &   Objem soustředěného odtoku rýhou \\
%  rillVolRest[m3]  &  $m^3$ &   Objem zbytku vody po soustředěném odtoku rýhou  \\
 surfaceFlow[m3/s]   &  $m^3_{-1}$ & Celkový průtok (plošný + soustředěný)  \\
 surfaceVolRunoff[m3]   &   $m$  & Celkový odteklý objem (plošný + soustředěný) \\
%  rillInflowVol[m3] & $m$ &  @@@ toto tam chcem? to je V\_inflow cast co jde do ryhy, pridal jsem to tam jednou kdyz jsem hledal nejakou chybu...\\
%  ratio & $m$ &  Počet krácení časového kroku v rýhách @@@(je pro nas?)\\
%  sheetCourantCrit & $m$ &  Courantovo kritérium pro plošná odtok @@@(je pro nas?)\\
%  rillCourantCrit & $m$ & Courantovo kritérium pro soustředěný odtok @@@(je pro nas?) \\
%  nIter & $m$ &  Počet iterací pří výpočty daného výpočetního kroku @@@(to bych tam nechal, muže to napověděl jestli se tam neděje něco moc rychle, což může znamenat chybu v zadaných datech, třeba dát 600 mm do srážky místo 60 mm) \\
  \hline
   \hline
   \multicolumn{3}{p{\textwidth}}{*výška hladiny u soustředěného odtoku není výška skutečné výška hladiny v rýze, ale v nadkritická výška hladiny vztažená na celou plochu výpočetní buňky}
 \end{tabular}

\end{table}


% % zakomentovane je do tmp



\begin{table}[h!]
 

 \centering
 \caption{Popis veličin  v {\tt.dat} souborech}
\label{tab:vystupytokdat}

% \begin{tabular}{p{4cm}lp{2cm}p{5cm}}
 \begin{tabular}{llp{0.5\textwidth}}
  \hline  \hline
 Název sloupce        & Jednotka     & Popis                                      \\ 
 \hline
 time[s]              &   $s$         &  Čas výpočetního kroku                    \\
 deltaTime[s]         &   $s$         &  Aktuální délka časového kroku            \\
 rainfall[m]          &  $m$          &  Srážková výška v aktuálním časovém kroku \\
 reachWaterLevel[m]        &  $m$          &  Výška hladiny plošného odtoku            \\
 reachFlow[m3/s]              &  $m^3s^{-1}$  &  Průtok plošného odtoku                   \\
 reachVolRunoff[m3]                  &  $m^3$     & Odteklý objem plošného odtoku     \\
%  reachVolInflow[m3]              &  $m^3$     & Suma přítoků z okolních buněk úseku v daném časovém kroku \\
%  reachVolRest[m3]         &  $m^3$     & Objem v úseku toku po odtoku      \\
  \hline
 \end{tabular}

\end{table}

% # Time[s];deltaTime[s];Rainfall[m];Waterlevel[m];V_runoff[m3];Q[m3/s];V_from_field[m3];V_rests_in_stream[m3]
% 
% 
% time[s]
% deltaTime[s]
% rainfall[m]
% reachWaterLevel[m]
% reachFlow[m3/s]
% reachVolRunoff[m3/s]
% reachVolInflow[m3]
% reachVolRest[m]
% zakomentovane je do tmp



\begin{table}[h!]
 

 \centering
 \caption{Popis veličin  v {\tt.dat} souborech}
\label{tab:vystupytokdat}

% \begin{tabular}{p{4cm}lp{2cm}p{5cm}}
 \begin{tabular}{llp{0.5\textwidth}}
  \hline  \hline
 Název sloupce        & Jednotka     & Popis                                      \\ 
 \hline
 time[s]              &   $s$         &  Čas výpočetního kroku                    \\
 deltaTime[s]         &   $s$         &  Aktuální délka časového kroku            \\
 rainfall[m]          &  $m$          &  Srážková výška v aktuálním časovém kroku \\
 reachWaterLevel[m]        &  $m$          &  Výška hladiny plošného odtoku            \\
 reachFlow[m3/s]              &  $m^3s^{-1}$  &  Průtok plošného odtoku                   \\
 reachVolRunoff[m3]                  &  $m^3$     & Odteklý objem plošného odtoku     \\
%  reachVolInflow[m3]              &  $m^3$     & Suma přítoků z okolních buněk úseku v daném časovém kroku \\
%  reachVolRest[m3]         &  $m^3$     & Objem v úseku toku po odtoku      \\
  \hline
 \end{tabular}

\end{table}

% # Time[s];deltaTime[s];Rainfall[m];Waterlevel[m];V_runoff[m3];Q[m3/s];V_from_field[m3];V_rests_in_stream[m3]
% 
% 
% time[s]
% deltaTime[s]
% rainfall[m]
% reachWaterLevel[m]
% reachFlow[m3/s]
% reachVolRunoff[m3/s]
% reachVolInflow[m3]
% reachVolRest[m]











% \subsection{State - typ průtoku na buňce}\label{sec:statpopis}
%   Jak bylo popsání no v kapitole~\ref{kap:tok} v modelu je možné řešit několik typů povrchového odtoku: plošný odtok, soustředěný odtoku a odtok hydrografickou sítí. Topografie hydrografické sítě je definována uživatelem. Vznik soustředěného odtoku je podmíněn překročením kritické výšky (popsáno v kapitole~\ref{sec:soustredenyodtok}). V programu jsou typy odtoku rozlišeny celočíselným identifikátorem označeným State, kde pokud State\\
% %   
%   \begin{tabular}{rcl}
%      =     &0&  dochází v buňce pouze k plošnému odtoku pokud \\
%      =     &1&  dochází v buňce k plošnému i soustředěnému odtoku  nebo pokud \\
%      =     &2&  @@@ plošný odtoku a rýha jen odtéká \\
%      && je to v teto verzi? Je to zajímavé pro uživatele? \\
%      $>$=  &1000&  je v buňce úsek hydrografické sítě. \\
%   \end{tabular}
% 
%   Identifikátor hydrografické sítě nemusí začínat číslem 1000 a nemusí být vzestupný (sestupný) u navazujících úseků. Tento identifikátor je v modelu definován jako 1000 + {\tt fid}, je tedy definován uživatelem nebo přiřazen použitým GIS softwarem. 
% 









%\section{Výstupní data} \label{section:vystupnidata}

%Po úspěšném ukončení modelu je do výstupního adresáře uloženo několik souborů. Každý z těchto souborů obsahuje hodnoty pro každou buňku rastru. Buňky, na kterých neprobíhal výpočet neobsahují žádné hodnoty, tedy NoData. Základní výstupy jsou uvedeny přímo ve zvoleném výstupním adresáři. Mimo hlavní výstupy jsou volitelně ukládány i dočasné výstupy sloužící pro případnou kontrolu. V podadresáři \textbf{temp} jsou dočasné soubory výpočtu v ploše a v podadresáři \textbf{temp_dp} jsou dočasné soubory vodních toků. \textbf{Dočasným výsledkům bude věnována jedna z dalších kapitol}



% 
% \pozn{
% 
% 
% \textbf{toto je origoš z DP}
% 
% \par Ne vždy se vytvoří všechny tyto výstupní soubory. Záleží na zvolených vstupních parametrech. Pokud uživatel nezadá žádnou bodovou vrstvu, nevytvoří se poslední textový soubor. 
% V případě, že uživatel nezvolí možnost soustředěného odtoku, nevytvoří se rastry a shapefile související s tímto typem odtoku. Rastr soustředění odtoku se nevytvoří při nezvolení vícesměrného odtoku. 
% Ostatní soubory se vytvoří pokaždé.  
% 
% \textbf{ z diplomky}
% 
% Výstupy se ukládají do adresáře nazvaného output. Cestu k němu si volí uživatel v rámci vstupních dat (viz kap. 2.3.1). Model prochází stále vývojem a dotýká se to i výstupních souborů. Princip ale zůstává stejný a jedná se spíše o úpravy zdrojového kódu zajištující lepší přehlednost a práci s kódem pro budoucí úpravy. Např. práce s vícerozměrnými maticemi a převedení všech výpočtů do základních (SI) jednotek. 
% Výsledkem modelu jsou soubory (.shp, .rst, .txt, .dbf), které reprezentují parametry (Zajíček J., 2014):
% hladina
% Výstupem jsou hodnoty maximální výšky hladiny pro každou buňku. Jedná se tedy o rastrovou vrstvu vytvořenou porovnáváním hodnot výšek hladiny v každém časovém kroku. Uložena je nejvyšší hodnota. Výška hladiny v jednotlivých krocích je získána pomocí bilance přítoků a odtoků do buňky.  
% průtok
% Výstupem jsou hodnoty maximálního průtoku pro každou buňku. Obdobně jako u hladiny jsou porovnávány hodnoty v jednotlivých krocích a uložena maximální hodnota. Hodnoty průtoku v jednotlivých časových krocích jsou vypočteny pomocí metody kinematické vlny (teorie viz kap. 1.5.2).
% infiltrace
% Výstupem infiltrace jsou hodnoty v každé buňce, které jsou během doby běhu modelu postupně načítány až do vyčerpání infiltrační kapacity.
% zbytkový objem
% Zbytkovým objemem se rozumí objem, který v dané buňce v časovém kroku zůstal. V případě odtoku veškeré vody z rastru je rastr nulový. Matematicky je objem vyjádřen jako rozdíl celkového objemu v buňce (zbytkový objem z předchozího kroku a přítoky) a povrchového a soustředěného odtoku.
% odtok
% Výstup týkající se odtoku slouží pro konečnou bilanci (kontrolu) a testování. Jedná se o celkové množství, které z buňky odteklo za celou dobu běhu modelu. 
% rychlost
% Rastr rychlostí je výstupem sloužící k určení erozní ohroženosti. Porovnávány jsou hodnoty skutečných rychlostí s limitními nevymílacími rychlostmi (viz tab. č. 3 ).
% napětí. 
% 
% Obdobou je rastr tečného napětí. Slouží k určení míst potencionálně nebezpečných. Hodnoty limitních hodnot tečného napětí jsou uvedeny ve stejné tabulce jako rychlosti průtok v rýze (viz tab. č. 3 ).
% 
% Průtok v rýze je rastrová vrstva znázorňující maximální průtok v rýze při soustředěném odtoku. Výstup je vytvořen jen při volbě typu výpočtu s uvažováním rýhového odtoku. Rýha vznikne pouze v buňkách, kde výška hladiny překročí hladinu kritickou. 
% rychlost v rýze
% Rastr obsahuje hodnoty maximální rychlosti v buňkách, kde je rýha vytvořena. Výpočet v rýhách probíhá odlišně oproti povrchovému odtoku. Jedná se o větší rychlosti, a proto na těchto buňkách probíhá výpočet za běžný časový krok 3x. V jiném případě by hrozilo, že výpočet nebude konvergovat.
% souhrn
% 
% Final evalution.txt je textový soubor, který obsahuje souhrn zadaných vstupů a čas běhu modelu a bilanci vody. 
% hydrogram
% Point hydrographs.txt je textový soubor s hodnotami výšky hladiny, průtoku, napětí, rychlostí v bodech zadaných vstupní bodovou vrstvou. Soubor slouží k tvorbě hydrogramů v těchto bodech. Automaticky je k vrstvě přidán bod, ve kterém je hodnota flow acumulation nejvyšší.
% Výstupem v současnosti je i řada dalších vrstev, které slouží ale spíše k tvorbě a testování modelu a pro samotného uživatele nejsou potřebné.	
% 
% }
% 
% 
% 
